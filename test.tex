\section{Цел на теста}

Целта на теста е проверка на придобитите знания и умения на ученици в 8-ми клас в ЕГ "Гео Милев" \-, гр. Добрич, свързани с подтемата "Приложни програми" от задължителната учебна програма, както и установяване на наличието на затруднения по разглежданата тема.

\section{Съдържателна рамка}

\subsection{Съдържателни области}

\begin{enumerate}
    \item Инсталация и деинсталация на софтуер
        
    - Инсталиране на приложни програми.
    
    - Деинсталиране на приложни програми.
    
    - Откриване и използване на нови приложения.
    
    \item Управление на софтуер и хардуер
    
    - Идентифициране на хардуерни и софтуерни проблеми.
    
    - Периодично актуализиране на операционната система и приложенията.
    
    \item Използване на помощни ресурси
    
    - Работа със самоучител при използване на нови програми.
    
    - Използване на помощна система при работа с непозната приложна програма.
    
    - Търсене на информация в помощната система на приложна програма.
    
    \item Компресиране и разкомпресиране на данни
    
    - Архивиране и разархивиране на данни.
    
    - Обяснение на процеса на компресиране и разкомпресиране на данни.
    
    - Работа със софтуер за архивиране и разархивиране на файлове.

\end{enumerate}

\subsection{Познавателни области}
\begin{enumerate}
    \item Знание и разбиране -
    
    - Разбиране на основни понятия, свързани с инсталация и деинсталация на софтуер.
    
    - Познаване на процеса на идентификация на хардуерни и софтуерни проблеми.

    - Осъзнаване на важността на актуализациите за сигурността на системата.

    \item Приложение -

    - Използване на технологични средства за инсталация и деинсталация на приложения.
    
    - Използване на помощни системи и самоучители за работа с нов софтуер.
    
    - Практическо приложение на методи за архивиране и разархивиране на данни.
    
    \item Анализиране и моделиране -
    
    - Анализиране на софтуерни и хардуерни проблеми и намиране на решения.
    
    - Моделиране на ситуации, изискващи актуализация на софтуер за повишаване на сигурността.

    - Разбиране и обяснение на процесите на компресиране и разкомпресиране на данни.
\end{enumerate}

\section{Структура на теста}

\subsection{Вид и времетраене}

Тестът е предвиден за полагане в писмен вид. Това ще се случи в рамките на един учебен час, чиято продължителност е 40 минути.

\subsection{Формат}
Тестът се състои от 16 задачи с избираем отговор. Към всяка задача са предоставени 4 възможни отговора, от които точно един е верен. 

Отбелязването на верен отговор се оценява с 1 точка. При отбелязан грешен отговор или при липсата на отбелязан такъв, съответната задача ще се оценява с 0 точки, т.е. не повлиява по негативен начин.

\section{Тестова спецификация}

\section{Инструкции за работа}
Тестът се състои от 10 задачи с избираем отговор. Всяка задача има 4 възможни отговора, от които точно един е верен. Отбелязването на верен отговор носи 1 точка, а отбелязването на грешен отговор или оставянето на задача без отговор носи 0 точки.

За да отбележите своя отговор, зачертайте със знака X буквата на избрания от Вас отговор. Ако решите, че първоначалният Ви отговор не е верен, запълнете кръгчето с грешния отговор и зачертайте със знака X буквата на друг отговор, който смятате за верен.

Време за работа – 40 минути.
Успешна работа!

\section{Тест}

\section{Ключ с верни отговори}

\section{Тестови карти на задачите}